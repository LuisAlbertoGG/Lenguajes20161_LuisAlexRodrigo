\documentclass[a4paper]{article}

\usepackage[english]{babel}
\usepackage[utf8x]{inputenc}
\usepackage{amsmath}
\usepackage{graphicx}
\usepackage[colorinlistoftodos]{todonotes}

\title{Universidad Nacional Autonoma de Mexico \\
 Facultad Ciencias\\
 Lenguajes de Programación 2016-1}
\author{ Alejandro Valdes Garcia  \\ Luis Alberto Godinez Galicia\\
  Rodrigo Rivera Paz}

\begin{document}
\maketitle



\section{Tarea 1}

1.
El siguiente es un esquema donde muestra la no-linealidad. \\
{with {x 2} \\
	{with{y 3} \\
		{with {z 4} \\
			{* x {+ y 1}}}}} \\
		
Se tienen los siguientes ambientes para el esquema anterior: \\
 \\
env0 = () \\
env1 = ((x 2) ()) \\
env2 = ((y 3) (x 2) ()) \\
env3 = ((z 4) (y 3) (x 2) ()) \\

Notemos que para obtener el valor de x, en la pila se tendría que bajar en el stack
hasta el env1, siendo asi 3 ambientes recorridos y para obtener y se bajaría hasta
el env2, recorriendo 2 ambientes, lo que no hace lineal su ejecición, sino cuadrática.
 \\
Utilizando la estructura de datos hash, se puede mejorar la complejidad, recibiendo así
un identificador y asi, usando las tablas hash se harían las busquedas de los valores.
Lo anterior haría que la complejidad mejorara a orden lineal O(n). \\

2. \\
  No \\
  {with {x 4} \\
    {with {x 7} \\
      {with {f {fun {y} {+ x y}}} \\
        {with {x 5} \\
          {f 10}}}}} \\
          
  En esta función no se cumple con la igualdad \\
 \\
3.

Forma de Bruijn
 \\

{with {5{fun {x} {fun {y} { + x y}}}3 \\
   {with {10{adder x}  \\
	 {<: 0  1> \\
	  {with{+ 10 <: 3 3>} {<: 1 1>0} \\
	    {+{+ <: 1 1><:1 0>} <: 3 2>}}}}}}} \\
	    
		Corridas:
		

\end{document}
