\documentclass{article}
\usepackage[utf8]{inputenc}

\title{Tarea 3}
\author{Rodrigo Rivera Paz \and Godínez Galicia Luis Alberto \and Valdés Galicia Alejandro}
\date{November 2015}

\usepackage{natbib}
\usepackage{graphicx}
\usepackage{anysize}
\marginsize{1cm}{1cm}{2cm}{2cm}

\begin{document}

\maketitle

\section{Problema I}
Haga el juicio de tipo para la función fibonacci y el predicado empty?
\\


\\
\newline
\textbf{l:list}


\rule[-2mm]{3cm}{0.1mm}
\\
\newline
\textbf{\Gamma\vdash l:list}


\rule[-2mm]{3cm}{0.1mm}
\\
\newline
\textbf{\Gamma\vdash(emptry? l):boolean}
\newline
\newline
\newline




\\
\newline
\textbf{\Gamm\vdash n:number \hspace{4mm}\Gamm\vdash 1:number\hspace{4mm}\Gamm\vdash n:number \hspace{4mm}\Gamm\vdash 2:number}

\rule[-2mm]{13cm}{0.1mm}
\\
\newline
\textbf{\Gamma(fib(number\rightarrow number)) \hspace{2mm} \Gamma\vdash (-n1):number \hspace{2mm}\Gamma(fib(number\rightarrow number))\hspace{2mm} \Gamma\vdash (-n2):number}


\rule[-2mm]{18cm}{0.1mm}
\\
\newline
\textbf{\Gamma\vdash n:number \hspace{2cm}\Gamma\vdash n:number \hspace{2mm}\Gamma\vdash 1:number \hspace{1cm} \Gamma\vdash(fib(-n1)):number\hspace{2mm} \Gamma\vdash(fib(-n2)):number}


\rule[-2mm]{17cm}{0.1mm}
\newline
\\
\textbf{\Gamma\vdash(zero? n):boolean \hspace{2mm} \Gamma\vdash1:number\hspace{2mm} \Gamma\vdash(=n1):boolean \hspace{2mm} \Gamma\vdash1:number \hspace{2mm} \Gamma\vdash(+ (fib(- n 1)) (fib(- n 2))):number}


\rule[-2mm]{12cm}{0.1mm}
\\
\newline
\textbf{\Gamma[n\leftarrow number]\vdash(cond [(zero? n)1] [(= n 1)1] [(+ (fib(- n 1)) (fib(- n 2)))]): number}


\rule[-2mm]{19cm}{0.1mm}
\\
\newline
bf{\Gamma[fib \leftarrow number]\vdash(lambda(n:number)):number\hspace{2mm}(cond [(zero? n)1] [(= n 1)1] [(+ (fib(- n 1)) (fib(- n 2)))]):number\rightarrow number}


\section{Problema II}
Considera el siguiente programa:
...
\\

\{[1]\} = \{(+\hspace{2mm}1\hspace{2mm}(first(cons\hspace{2mm} true \hspace{2mm}false)))\}\hspace{2mm}=\{(+ [2] [3])\}=number\hspace{2mm} y \{[2]\}=\{[3]\}

\{[2]\} = number

\{[3]\} = \{(first(cons\hspace{2mm} true\hspace{2mm} empty))\}=number \hspace{2mm} y \hspace{2mm} \{[4]\}=list

\{[4]\} = \{(cons\hspace{2mm} true\hspace{2mm} empty)\}=list y \hspace{2mm}\{[true]\}=list, \hspace{2mm}\{[empty]\}=list

\{[5]\} = \{true\}=boolean !
\textbf{La operación cons tiene dos argumentos y ambos deben de ser de tipo list, pero a la hora de hacer las restricciones vemos que true es de tipo boolean, lo cual debe mandar un error.}



\section{Problema III}
Considera la siguiente expresióon con tipos:
...
\\

\{[1]\} = \{f\} \rightarrow \{[1]\}
\\
\{[2]\} = \{x\} \rightarrow \{[2]\}
\\
\{[3]\} = \{y\} \rightarrow \{[3]\}
\\
\{[4]\} = \{(cons\hspace{2mm} x \hspace{2mm} (f (f \hspace{2mm}y))\} \hspace{2mm} con \hspace{2mm}\{x\}=list \hspace{2mm} y \hspace{2mm} \{(f (f \hspace{2mm}y))\}=list
\\
\{[x]\} = number
\\
\{[6]\} = \{(f (f \hspace{2mm}y))\} = list \hspace{2mm} con \hspace{2mm}\{f\}=number \hspace{2mm} y \hspace{2mm} \{[7]\}=list
\\
\{[7]\} = \{(f \hspace{2mm}y)\} = list \hspace{2mm} con \hspace{2mm}\{f\}=number \hspace{2mm} y \hspace{2mm} \{y\}=list
\\
\textbf{Gracias a estas restricciones podemos inferir los tipos de los Cn's.}

\section{Problema IV} 
\textbf{no cambian,por que sirven para revisar que los tipos sean correctos en la funcion } 
\section{Problema V}
-ventajas explisito \\
reutilizacion de codigo, mantenimiento\\
-implícito\\
cambiar su tipo sin hacer un cast \\
El polimorfismo puede hacerse con referencias de superclases abstract, super-clases normales e interfaces.\\


-desventajas
-explícito
mucho codigo 
el tipo de la referencia (clase abstracta, clase base o interface) limita los métodos que se pueden utilizar y las variables miembro a las que se pueden acceder.

-implisito
no saber que tipo es 




\section{Problema VI}
general \\
-ventajas\\
bibliotecas\\ 
resuelve diferente tipos de problemas \\
-desventajas\\
el algunos problemas son lentos\\
difícil de abstraer \\
\\
\\

específico\\
\\
-ventajas\\
 es fácil de aprender (algunos)\\
 rápido \\
seguridad \\
-desventajas\\
solo sirve para una cosa\\
\\

sql\\
\\
fue creado para tener datos en tablas y regresa tablas\\
regresa la tabla nombres en una columna y su calificacion
\\
\\
SELECT nombre,calificación\\
FROM estudiante\\
\\
css creado solo para la parte visual de una página\\
\\
\begin{verbatim}

/*CSS sobre selector de identificador*/\\
#header {
       background-color: #ff0000;
       color: #ffffff;
       font-size: 26px;
       
}
\end{verbatim}
\\html \\
para dar la solo estructura a la página web 
\\
 \begin{verbatim}

<!DOCTYPE html PUBLIC "-//W3C//DTD XHTML 1.0 Transitional//EN"
  "http://www.w3.org/TR/xhtml1/DTD/xhtml1-transitional.dtd">
<html xmlns="http://www.w3.org/1999/xhtml">
<head>
<meta http-equiv="Content-Type" content="text/html; charset=iso-8859-1" />
<title>Ejemplo documento</title>
</head>
<body>
<p>Un párrafo de texto.</p>
</body>
</html>

\end{verbatim}


\end{document}
